\documentclass{article}
\usepackage{FinalYearProjectReport}

% packages for references
\usepackage{cite}
\usepackage{url}


% uncomment this line to double line spacing for proof reading
% \linespread{2}

% packages and settings for graphics
\usepackage[pdftex]{graphicx}
\graphicspath{{./}}
\DeclareGraphicsExtensions{.png}
\usepackage[final]{pdfpages}


\title{Traffic Reporter - Final Report}
\name{Michael Little}
\address{Department of Electrical and Computer Engineering\\
University of Auckland, Auckland, New Zealand}


\hyphenation{and-roid}


\begin{document}

\begin{titlepage}


\vspace*{15em}


\centering

{\LARGE
Department of Electrical \& Computer Engineering \\
Final Year Research Project 2013, Final Report}

\hspace{2em}

% notes on latex tables use "&" as colum sperator
\begin{table*}[h]
\centering
\begin{tabular}{ll}
Project Title: & Traffic Information Engine for Auckland Traffic Lights \\
Project Number: & 14 \\
Supervisor Name: & Dr. Nasser Giacaman \\
Second Examiner Name: & Dr. Partha Roop \\
Your Name: & Mike Little \\
Your UID: & 2626904 \\
Partner Name: & Andrew Luey \\
Date submitted: & 15 September 2013 \\

\end{tabular}
\end{table*}
\begin{table}


\end{table}
\pagebreak

\vspace*{25em}

{\Large Declaration of Originality}

\hspace{5em}

This report is my own unaided work and was not copied from 
nor written in collaboration with any other person.

Name: Mike Little


\end{titlepage}




\maketitle



\begin{abstract}
This report describes the initial work and current
progress of a project to create a new version of software that
analyses traffic volume through intersections for Auckland
Transport.

\end{abstract}



\section{Introduction}

Auckland Transport uses a system that collects the traffic
volume called SCATS (Sydney Coordinated Adaptive Traffic
System)\cite{sims1981scat}. It uses the volume information it collects to
intelligently change the phase lengths of traffic lights. It also
tries to coordinate sets of traffic lights along a route so that
cars going in a certain direction, towards the CBD for
example, will get green lights all the way through. It will also
drop phases if there aren’t any cars sitting on a detector so that
other cars won’t have to wait.

The alternative to an adaptive system like SCATS is to have
fixed phase lengths at every intersection. Using an adaptive
system has many benefits including reduced travel time,
reduced fuel usage, and reduced air pollution \cite{sims1981scat}. With travel
times SCATS consistently matches or outperforms static
traffic systems \cite{hunter2012probe}.

Problems arise in an adaptive system like SCATS when the
detectors become faulty which happens fairly often. Faulty
detectors can report anything from zero volume to an
impossibly high amount. These false values throw off the
scheduling of phases.

Andrew Luey and myself were tasked by Auckland
Transport (AT) to recreate a piece of their existing software
that they use to analyze traffic volumes through an
intersection. The application can look at the volume
information to identify possibly malfunctioning detectors as
well as identify trends in traffic flow. It is also commonly used
to generate reports on traffic volumes on main roads, for
example average traffic volume over a month, the busiest
intersections, busiest routes, etc.

Their existing piece of software is old (written in the
1980’s) and has limitations in what it can do as well as
suffering from a few usability issues that are symptomatic of
when it was written.
To recreate the application we began by doing research into
the appropriate platform for the new application because the
client expressed an interest in being able to eventually view
volume information on a smartphone or tablet. We became
familiar with the existing application to see what functionality
we had to duplicate, as well as to see what features were
lacking or could be improved upon.
Currently we are in the middle of implementing the new
application using C\# and the Microsoft Windows Presentation
Foundation (WPF) [4]. We have created a basic user interface
and written the code to read the files, which contain volume
information for an intersection.

\section{Research}
\subsection{Existing Application}
To recreate the existing application, we had to become
familiar with it. We explored its features and what we liked
and disliked about it.

The current application has two modes; essentially it is two
programs in one. One mode displays information on traffic
volume (the VS mode), the other displays information on the
traffic light phasing. Because of scope constraints we looked
only into the traffic volume part of the application. Later if
time permits, we might look into recreating the phasing part of
the application.

The VS mode has four different views; a graph view, a
table view, a column view, and a dump view. The two most
commonly used views are the graph and table view. We were
unable to ascertain what the column view is actually used for.
The dump view gives a human readable representation of a
volume store file that is very close to how it is encoded. This
was useful to us when we were learning how to read the files,
but is unused otherwise.

The graph view shows a line graph with time on the X-axis
and traffic volume on the Y-axis. Each detector at an
intersection can have it’s own series or they can be grouped
together. The time interval can be set between 5 minutes and
an hour. Data is collected every 5 minutes, so if you increase
the interval it plots the average.

The graph view is useful for seeing trends in traffic flow.
For example for some detectors there is a large spike between
6am and 7am and then it falls off over the rest of the day. We
can look at this and see that that detector probably heads
toward the CBD.

The table view displays volume information for 5-minute
intervals for each hour with hourly totals in the bottom row.
(See Fig.1) There is a table for both AM and PM traffic with
the total for each half of the day and the peak amount and
time. Each approach has its own pair of tables, so like the
graph view you can group detectors together to get an
indication of volume for a specific direction. You can change
the interval to get the totals over that interval.
The table view is good for finding specific values at
specific times, but is poor for identifying trends.
The column view is used only to export data to an Excel
spreadsheet for further manipulation.

\subsection{Requirements Gathering}
For this project we have had weekly or fortnightly
meetings with our client. To begin with these meetings were
used to gather requirements and to find out more about how
the existing application works and what it is used for.
The main use of the existing application is to use the
volume data for reports. For example, finding the busiest
intersections for a given month, or finding the busiest time of
day for different intersections, etc.

We also went to the Auckland Transport office in the
Auckland CBD. We saw that it is used in tandem with another
program that directly interfaces with SCATS. The other
program shows detectors being triggered in pseudo real time
and shows planned and actual light phasing. Also available to
them were camera feeds of some of the main intersections.
Using all of these tools they could identify an intersection
whose phasing is incorrect or non-optimal and then check the
volumes of those intersections to see if there are any obviously
faulty detectors.

The visit to Auckland Transport was instrumental in our
understanding of what it is we are trying to replicate and
improve. We were able to identify significantly more use
cases, from seeing the application in action in the environment
that it is used in.

\subsection{Platform Choice}
One of the requirements that the client stressed to us was
future extensibility of platforms. To make the application
portable we initially suggested making a web based
application with a RESTful API [1]. This way it would be
very easy to introduce new platforms that would only have to
make use of the API.

The other choice was a windows desktop application that
loads volume files locally. The main advantage of this
platform choice would be that there would be significantly less
architectural work involved and could be developed more
rapidly. (See Table 1 for pros and cons).

We felt it would be most appropriate to create the
application as a web service. One of the key criteria expressed
by Auckland Transport being extensibility, we felt that a web
service meets this criteria better. Auckland Transport sees this
project as something that is likely to be extended and
improved over the years. Using a web service promotes client
platform in-dependence, which allows Auckland Transport to
reuse the same program for other platforms such as mobile
applications.

Ultimately the client decided to go with a standalone
windows application. This was due to a few factors, the main
ones being that a web service would take longer to create and
that AT was reluctant to grant us access to the infrastructure
required to host such a service.

\section{Current Progress}
\subsection{Reading the Volume Information}
The first thing we did after gathering requirements was to
try to decode the volume information. The data is in binary
and uses an encoding that aims to reduce file size. (See Figure
2 for format specification).
Figure 1 Volume information Figure 1 Table view of existing application data format
We ran into a few small issues initially when trying to
interpret the information; the main one being that information
encoded across multiple bytes was little endian.
We solved most of our issues like this by printing out a hex
dump and using pens and highlighters to manually interpret
the binary information.

\subsection{Designing the User Interface}
In designing the user interface (UI) for the application we
decided to create lo-fi prototypes to quickly create a rough
view of our proposed UI (See Appendices) These were very
useful for being able to propose UIs to our client and have him
critique them without us having to spend a lot of time
developing a full blown UI and then have it be rejected.

\subsection{Application Implementation}
After we went over the lo-fi prototypes with our client and
made the changes he required we began to implement the
actual application using the Microsoft Windows Presentation
Foundation (WPF)[4]. WPF uses XML files to declaratively
define a UI and C\# to imperatively bind them to data and
perform related logic.

My partner and I decided to split the work at this point; I
would write the code to read the volume information into
memory and provide an interface for getting specific
information required for any reporting, while he would realize
our lo-fi prototypes and create the beginnings of our UI.
To load the volume information I initially wrote a singleton
[6] class called VolumeStore that loaded all the information
from a hard coded file. It did all of the decoding itself and
stored information in a few Dictionaries [5]. Later I changed
this and created the abstraction of a Record and a pseudo
factory [6] that would create a Record given a byte array. Now
the VolumeStore class is less concerned with decoding the
binary data and more concerned with representing it in a way
that is useful to the other components of our application.
Once my partner had created the basics of a UI, I added a
menu item that would let you specify which file you wanted to
load volume information from. Loading the file and decoding
it is a long running task and initially I was doing it on the UI
thread, thus blocking the UI and making it unresponsive. The
C\# standard library provides a class for performing long
running tasks called BackgroundWorker [7], but, I also wasn’t
providing any feedback to the user so they would have no
indication of when the loading had been completed.
Fortunately, I found a small library called ProgressDialog [3]
that solved both of these issues.

Once we had a basic UI that could load volume
information, we needed a way to display it. My partner and I
decided that I would recreate the graph view of the original
application and he would recreate the table view. To display
the data I found a WPF charting library that could display line
charts. There were a number of very nice looking libraries that
you could pay for, but I went for a free one called Dynamic
Data Display (D3) [2]. With it I have been able to display the
line graph shown in Figure 3.

\section{Future Work}
\subsection{Graph View}
Currently the graph view only displays a series for the first
detector of the first intersection available from the
VolumeStore instance it receives. So there needs to be some
sort of configuration information made available to it.

Also, if no data has been loaded into the VolumeStore
instance, the program will exit with an exception. The
program should either tell the user that they haven’t loaded
any volume data, or make it impossible for the graph view to
be displayed.

\subsection{Table View}
At the moment the table view displays an empty table and
so needs to be populated. The approach used to get
information to the graph view can be reused here to get an
idea of what’s involved in using data tables in WPF.
Like the graph view we hope to add some sort of
highlighting to the table to indicate useful statistics, for
example, the highest volume detector, the busiest time, etc.

\subsection{Report Configuration}
In order to recreate the existing functionality of grouping
detectors into approaches we will implement a screen/view
similar to Appendix 2. The idea is to provide a drag and drop
interface for grouping detectors with additional buttons to
clear all approaches and to distribute each detector to its own
approach.

This screen will produce some sort of Configuration object
that we will be able to pass to the graph and table views and
let them display the data appropriately.

\subsection{Excel Export}
One of the things that the existing application is used for is
to simply load the files so that the data can be copied to an
Excel spreadsheet for doing numerical analysis. We are
aiming to make this workflow a little smoother by adding the
ability to export data to a CSV file already formatted, thus
easing the process of manual copying pasting.

\subsection{Preloading of VS store files}
Another feature that we’d like to add is to have a directory
that the application will look into to find VS files and load
them before the application starts. A splash screen with a
progress bar can be shown so that the user knows that the
application is starting.

This means that we can do the long load before the user
begins interacting with the application, so when they start
viewing tables and graphs it’s nice and quick. We feel that
having a “big load upfront” will help with the applications
perceived performance.

Importing files from another location will copy them to that
directory as well as loading them into a VolumeStore instance,
so next time they run the application they won’t have to
import the file again.

This will be a good opportunity to load configuration files
that store saved report configurations as well.

\section{Conclusions}
The advantages of an adaptive traffic system like SCATS
are many. However it does not solve all problems. Traffic data
must be collected and examined to make informed decisions
regarding Auckland’s transportation infrastructure. The
programs used to view and analyze this data must be effective
at displaying this data and helping their user identify and
extract meaningful information from it.

I have outlined the current progress and future work in
recreating Auckland Transport’s volume reporter. The purpose
of the recreation being making the tool more usable and more
effective in displaying volume data to its user so that
Auckland Transport can make better decisions as to how it
improves and maintains Auckland’s roads.


\bibliographystyle{IEEEtran}
\bibliography{final_report}


\end{document}