\documentclass{article}
\usepackage{FinalYearProjectReport}

% packages for references
\usepackage{cite}
\usepackage{url}


% uncomment this line to double line spacing for proof reading
% \linespread{2}

% packages and settings for graphics
\usepackage[pdftex]{graphicx}
\graphicspath{{./}}
\DeclareGraphicsExtensions{.png}
\usepackage[final]{pdfpages}


\title{Traffic Reporter - Final Report}
\name{Michael Little}
\address{Department of Electrical and Computer Engineering\\
University of Auckland, Auckland, New Zealand}


\hyphenation{and-roid}


\begin{document}

\begin{titlepage}


\vspace*{15em}


\centering

{\LARGE
Department of Electrical \& Computer Engineering \\
Final Year Research Project 2013, Final Report}

\hspace{2em}

% notes on latex tables use "&" as colum sperator
\begin{table*}[h]
\centering
\begin{tabular}{ll}
Project Title: & Traffic Information Engine for Auckland Traffic Lights \\
Project Number: & 14 \\
Supervisor Name: & Dr. Nasser Giacaman \\
Second Examiner Name: & Dr. Partha Roop \\
Your Name: & Mike Little \\
Your UID: & 2626904 \\
Partner Name: & Andrew Luey \\
Date submitted: & 15 September 2013 \\

\end{tabular}
\end{table*}
\begin{table}


\end{table}
\pagebreak

\vspace*{25em}

{\Large Declaration of Originality}

\hspace{5em}

This report is my own unaided work and was not copied from 
nor written in collaboration with any other person.

Name: Mike Little


\end{titlepage}




\maketitle



\begin{abstract}
This report describes the initial work and current
progress of a project to create a new version of software that
analyses traffic volume through intersections for Auckland
Transport.

\end{abstract}



\section{Introduction}

Auckland Transport uses a system that collects the traffic
volume called SCATS (Sydney Coordinated Adaptive Traffic
System)\cite{sims1981scat}. It uses the volume information it collects to
intelligently change the phase lengths of traffic lights. It also
tries to coordinate sets of traffic lights along a route so that
cars going in a certain direction, towards the CBD for
example, will get green lights all the way through. It will also
drop phases if there aren’t any cars sitting on a detector so that
other cars won’t have to wait.

The alternative to an adaptive system like SCATS is to have
fixed phase lengths at every intersection. Using an adaptive
system has many benefits including reduced travel time,
reduced fuel usage, and reduced air pollution \cite{sims1981scat}. With travel
times SCATS consistently matches or outperforms static
traffic systems \cite{hunter2012probe}.

Problems arise in an adaptive system like SCATS when the
detectors become faulty which happens fairly often. Faulty
detectors can report anything from zero volume to an
impossibly high amount. These false values throw off the
scheduling of phases.

Andrew Luey and myself were tasked by Auckland
Transport (AT) to recreate a piece of their existing software
that they use to analyze traffic volumes through an
intersection. The application can look at the volume
information to identify possibly malfunctioning detectors as
well as identify trends in traffic flow. It is also commonly used
to generate reports on traffic volumes on main roads, for
example average traffic volume over a month, the busiest
intersections, busiest routes, etc.

Their existing piece of software is old (written in the
1980’s) and has limitations in what it can do as well as
suffering from a few usability issues that are symptomatic of
when it was written.
To recreate the application we began by doing research into
the appropriate platform for the new application because the
client expressed an interest in being able to eventually view
volume information on a smartphone or tablet. We became
familiar with the existing application to see what functionality
we had to duplicate, as well as to see what features were
lacking or could be improved upon.
Currently we are in the middle of implementing the new
application using C\# and the Microsoft Windows Presentation
Foundation (WPF) [4]. We have created a basic user interface
and written the code to read the files, which contain volume
information for an intersection.

\section{Conclusions}
You might want some of these.


\bibliographystyle{IEEEtran}
\bibliography{final_report}


\end{document}